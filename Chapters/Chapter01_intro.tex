%************************************************
\chapter{Introduction}\label{ch:intro}
%************************************************
% TODO: DANI suggests to remove empty chapter page. Add image or \chapter -> \section
% switch between these depending on page size, or modify directly
% same for the rest of the chapters
%\tikz[remember picture,overlay] \node[opacity=0.3,inner sep=0pt] at (current page.center){\includegraphics[width=\paperwidth,height=\paperheight]{./Figures/cover/sesshu_1.jpg}};
%\tikz[remember picture,overlay] \node[opacity=0.3,inner sep=0pt] at ([yshift=3cm,xshift=2cm]current page.center){\includegraphics[width=\paperwidth,height=\paperheight]{./Figures/cover/sesshu_1.jpg}};
% \clearpage

\section{Life Histories and Behavior}

Concern over the loss of biodiversity associated with human-induced rapid 
environmental changes has generated an urgent need to understand why organisms 
differ in their response to environmental changes. 

At the most fundamental level, the persistence of population hinges on the fate 
of individuals surviving and reproducing in their environments. If individuals 
are able to reproduce at a higher rate than they die, the population will 
increase in numbers and can eventually become established and spread; if the 
balance is negative, however, the population will decrease over time and end up 
extinct. Because the rates of birth and death are ultimately determined by how 
organisms allocate their limited time and energy to reproduction and survival 
\citep{stearns1992evolution}, life history theory has long been 
deemed essential to understanding the dynamics of populations 
\citep{Saether2004, Sol2012a}.

However, as I argue in this thesis, if we want to fully understand how life 
history affects the population dynamics of animals exposed to environmental 
changes, we need to explicitly consider the role of behaviour. The argument for 
the need to better integrate behaviour into life history theory is founded upon 
three main principles. The first is the fact that behavioural responses are part 
of the adaptive machinery of animals to cope with uncertainties and evolutionary 
disequilibria of novel environments. While the idea is not new \citep{mayr1965}, 
recent theoretical and empirical advances provide a strong foundation for 
moving forward \citep{Sol2020, Ducatez2020}. The second argument is the growing 
evidence that behaviour affects and is affected by life history, which implies 
that both are part of a same adaptive strategy \citep{Sol2016, Sol2016a}. Thus, 
when we examine how life history affects population dynamics, including 
extinction or colonization, we are considering not only life history mechanisms 
but also mechanisms related to behavioural responses to novel environments 
\citep{Sol2016}. The last argument is that behaviour mediates some life history
mechanisms of response to novel environments, particularly those related to
environmental uncertainty and adaptive mismatch. By clearly delineating these
mechanisms, we can better infer when it is necessary to consider behaviour.
Altogether, the above principles create a new way to understand how life history
influences population growth in novel or changing environments, potentially 
contributing to a more predictive theory. Such a theory may be useful to help 
prevent and mitigate the ecological and economic impact of biological invasions 
\citep{Kolar2002, Vall-llosera2009, Leung2012}. The new theory should also be of 
great importance in predicting extinction risk associated with human-induced 
rapid environmental changes like habitat destruction and climate change 
\citep{Saether2000, Sih2011}.

\clearpage

\begin{small}
\begin{mdframed}
\subsection*{Box 1: Life History}
Life history strategies are the different ways to though which organisms 
allocate the limited resources among different components of the fitness such as 
reproduction, survival and development \citep{stearns1992evolution,roff2002}⁠. 
Each strategy is defined by a combination of phenotypic traits with direct 
effects on the fitness such as clutch size, broods per year, age at first 
breeding or lifespan \citep{Violle2007}⁠.
Mechanisms that generate the trade-offs explaining the observed covariance among 
traits include resource partitioning, correlational selection between traits 
and antagonistic pleiotropy \citep{Roff2007, Stearns1989a}. 
Incompatible physiological states mediated by the endocrine system 
\citep{Ricklefs2002}⁠ generates another source of mechanisms linking life 
history traits mediated by behavior \citep{Reale2010a}.

There is a consensus on the general features of a plausible explanation of the 
evolution of life history traits \citep{Stearns2000}⁠: (1) life histories 
are shaped by the interaction of extrinsic and intrinsic factors, (2) the 
extrinsic factors are ecological impacts on survival and reproduction; (3) the 
intrinsic factors are trade-offs among life history traits and lineage-specific 
constraints on the expression of genetic variation.
\end{mdframed}
\end{small}


\section{Objectives}

In my thesis I addressed fundamental unresolved questions about the 
interaction of life history and behavior in facilitating or impeding the 
response to rapid environmental changes. Chapter \ref{ch:LHaxes} describes the 
main axes of life history variation in birds. Chapters \ref{ch:LH-Behaviour 
model} and \ref{ch:POLS} explore the links between life history and behavior, 
the first using a theoretical model focused on the process of colonization of 
novel environments, and the second analysing empirical data from urban and 
non-urban populations looking for life histories and behavior patterns. The 
specific goals of the chapters are:

\begin{itemize}
\item \textbf{Chapter \ref{ch:LHaxes}: To describe the axes of life history 
variation in birds}\\ \\

A major axis of life history variation is the so-called fast-slow continuum, 
which reflects a fecundity-survival trade-off. However, defining and 
quantifying the fast-slow axis has proven to be difficult because there is no 
consensus regarding the life history traits that best define it. A way to 
address this problem is to use a demographic approach that identifies the 
combination of traits that best describe the underlying trade-off. 
In this chapter, I defined the fast-slow axis that better predicts the 
elasticity of the adult survival and generation time from available demographic 
models, and describe other less studied axes of life history from the remaining 
variation. Then, I generated a a global dataset for birds with the position of 
each species along the new fast-slow axis, which may be used for comparative 
analyses (see Chapter \ref{ch:POLS}).

\item \textbf{Chapter \ref{ch:LH-Behaviour model}: To explore the mechanisms 
linking life history and behavior}\\ \\

I developed a theoretical model simulating the introduction of a species in a 
new environment and evaluate how the life history and behavior could interact 
and affect the persistence of the population under stochastic and maladaptive 
scenarios.

\item \textbf{Chapter \ref{ch:POLS}: To analyse the effects of life history and 
behavior in the ability to colonize urban habitats}\\ \\
By means of comparative analysis, I describe patterns linking life history and 
behavior and how they affect the colonization of urban areas in birds.
\end{itemize}
