%************************************************
\chapter[Appendix 3.1: Chapter 3 - Model description and parameterization]{
Appendix 3.1: Chapter 3 - Model description and parameterization}
\label{ch:Appendix3.1}
%************************************************
\renewcommand{\thefigure}{A.3.1.\arabic{figure}}
\setcounter{figure}{0}

\renewcommand{\thetable}{A.3.1.\arabic{table}}
\setcounter{table}{0}

\section*{Description of the stochastic population model}\label{model-description}

At any temporal point in a given simulation run the population is described by a
cohort of juveniles, sub-adults at all stages, non-breeding adults (i.e. those
that skip breeding) and failed and successful adult breeders. Those individuals
may belong to one of the two habitats that typify our simulated scenario. Growth
from one stage to the next one, shift between habitats and the birth and
establishment of new individuals are all computed by stochastically drawing from
corresponding binomial distributions. See Figure \ref{fig:figApp3.2.1} for a graphic overview of
all possible state transitions and Table \ref{tab:tabApp3.1.1} for parameters definition.

\begin{enumerate}
  \item The first calculation, for a given temporal step $t$ and for each
habitat, is to calculate how many full adults will become non-breeders and how
many will be (successful or failed) breeders:
    \begin{equation*}
      A_{h,t}^{b} = Bin \left( {A}_{h,t} , {p}_{h}^{b} \right)
    \end{equation*}

  \item Then, we enter the breeding algorithm, which runs recursively $m$ times
within a loop. Within this loop only juvenile recruitment and survival, as well 
as habitat shift for successful and failed adult breeders are evaluated.
  \begin{enumerate}
    \item Within the loop, at the beginning of a given brooding event $i$, the
number of adults that breed successfully is decided for each habitat by randomly
drawing from a binomial distribution:
      \begin{equation*}
        A_{h,t}^{b_{s}} = Bin \left( {A}_{h,t}^{b} , 1 - {p}_{h}^{b_{f}} \right)
      \end{equation*}
    \item That, in turn, allows us to calculate the number of adults that fail 
to breed:
      \begin{equation*}
        A_{h,t}^{b_{f}} = A_{h,t}^{b} - A_{h,t}^{b_{s}}
      \end{equation*}
    \item Then, the number of juveniles that are bred and survive until the 
next year are updated at each loop step as:
      \begin{equation*}
        {J}_{h,t} = {J}_{h,t} + Bin \left( {A}_{h,t}^{{b}_{s}} \cdot q, {p}_{h, 
{S}_{j}} \right)
      \end{equation*}
    \item Next, after hatching has taken place, part of the successful or 
failed adult population may move between habitats. We model those actions by 
stochastically drawing from a binomial distribution the number of adults that 
change habitat:
      \begin{equation*}
{\Delta}_{1\rightarrow2}^{x} =Bin \left({A}_{1,t}^{x} , {p}_{1\rightarrow2}^
{x} \right)
      \end{equation*}
      \begin{equation*}
{\Delta}_{2\rightarrow1}^{x} =Bin \left({A}_{2,t}^{x} , {p}_{2\rightarrow1}^
{x} \right)
      \end{equation*}
Where, in this case, $x \in \{ {b}_{s} , {b}_{f} \}$.
    \item Then, the number of successful and failed breeders will be updated:
      \begin{equation*}
{A}_{1,t}^{x} = {A}_{1,t}^{x} - {\Delta}_{1\rightarrow2}^{x} +
{\Delta}_{2\rightarrow1}^{x}
      \end{equation*}
      \begin{equation*}
{A}_{2,t}^{x} = {A}_{2,t}^{x} - {\Delta}_{2\rightarrow1}^{x} +
{\Delta}_{1\rightarrow2}^{x}
      \end{equation*}
    And thus, with the new calculation:
      \begin{equation*}
{A}_{1,t}^{b} = {A}_{1,t}^{{b}_{s}} + {A}_{1,t}^{{b}_{f}}
    \end{equation*}
    \begin{equation*}
{A}_{2,t}^{b} = {A}_{2,t}^{{b}_{s}} + {A}_{2,t}^{{b}_{f}}
      \end{equation*}
    Then the algorithm goes back to point a) above $m$ times, after which it 
jumps from e) above to point 3 just below.
  \end{enumerate}
  \item When the simulation includes sub-adults (i.e. age of first reproduction 
\textgreater{1}), we must account for the fact that they may also move between habitats:
    \begin{equation*}
{\Delta}_{1\rightarrow2}^{r} =Bin \left({S}_{1,t}^{r} , {p}_{1\rightarrow2}^{r}
\right)
    \end{equation*}
    \begin{equation*}
{\Delta}_{1\rightarrow2}^{r} =Bin \left({S}_{2,t}^{r} , {p}_{2\rightarrow1}^{r}
\right)
    \end{equation*}
  \item Moreover, sub-adults are also affected by survival, which is modeled by 
stochastically drawing from a binomial distribution:
    \begin{equation*}
{S}_{1,t+1}^{r} =Bin \left({S}_{1,t+1}^{r} - {\Delta}_{1\rightarrow2}^{r} +
{\Delta}_{2\rightarrow1}^{r} , {p}_{1, {S}_{sa}} \right)
    \end{equation*}
    \begin{equation*}
{S}_{2,t+1}^{r} =Bin \left({S}_{2,t+1}^{r} - {\Delta}_{2\rightarrow1}^{r} +
{\Delta}_{1\rightarrow2}^{r} , {p}_{{2,S}_{sa}} \right)
    \end{equation*}
  \item Then, we also allow non-breeders to change habitats by stochastically 
drawing from a binomial distribution:
    \begin{equation*}
{\Delta}_{1\rightarrow2,t}^{nb} = Bin \left({A}_{1,t}^{nb},
{p}_{1\rightarrow2}^{nb} \right)
    \end{equation*}
    \begin{equation*}
{\Delta}_{2\rightarrow1,t}^{nb} = Bin \left({A}_{1,t}^{nb},
{p}_{2\rightarrow1}^{nb} \right)
    \end{equation*}
  \item Consequently, the population at $t+1$ of those non-breeding adults must 
be updated as follows:
    \begin{equation*}
{A}_{1,t+1}^{nb} = {A}_{1,t}^{nb} - {\Delta}_{1\rightarrow2}^{nb} +
{\Delta}_{2\rightarrow1}^{nb}
    \end{equation*}
    \begin{equation*}
{A}_{2,t+1}^{nb} = {A}_{2,t}^{nb} - {\Delta}_{2\rightarrow1}^{nb} +
{\Delta}_{1\rightarrow2}^{nb}
    \end{equation*}
  \item Next, we account for survival probability of all types of adults:
    \begin{equation*}
{A}_{h,t+1}^{x} =Bin \left({A}_{h,t+1}^{x} , {p}_{h, {S}_{x}} \right)
    \end{equation*}
  where, in this case, $x \in \{ {b}_{s}, {b}_{f}, nb \}$. Values
${A}_{h,t+1}^{x}$ inside the binomial correspond to those at the end of the
breeding loop when $x=b_{s}$ or $x=b_{f}$.
  \item Total adult population is then:
    \begin{equation*}
{A}_{h,t+1} = \sum_{x \in \{ {b}_{s}, {b}_{f}, nb \}} {{A}_{h,t+1}^{x}}
    \end{equation*}
  \item Finally, populations are updated simply by moving up one stage and 
juveniles become adults or sub-adults according to the age of first 
reproduction.
\end{enumerate}


\section*{Exploration of the parameter space}\label{model-exploration}

The range of demographic parameters comes from empirical data from birds. With 
the chosen parameters we estimated juvenile survival $p_{1,s}$ for a given 
deterministic growth rate $\lambda$ corresponding to the Leslie matrix model by
solving the Euler-Lotka equation:

\begin{equation*}
{p}_{1, {S}_{j}} = \frac {{{p}_{1, {S}_{sa}}} ^ {1-AFR} \left({\lambda}^{1+AFR}
- {p}_{1, {S}_{b}} \cdot {\lambda} ^ {AFR} \right)} {q \cdot \lambda}
\end{equation*}

where $AFR$ is the age at first reproduction and other notation follows Table
\ref{tab:tabApp3.1.1}. Once demographic parameters in habitat 1 are defined we
modify the habitat 2 parameters according to the different scenarios. In
Scenario 1, the parameters in habitat 2 are the same than in habitat 1. In
Scenario 2 we increase adult and subadult mortality $n$ times (1.5 in Scenario
2.1 and 2 in Scenario 2.2). To increase $p$ probabilities $n$ times we apply
$p^{n}$ and therefore, adult survival in habitat 2 is:
\begin{equation*}
{p}_{2, {S}_{b}} =1- {\left(1- {p}_{1, {S}_{b}} \right)}^{{1} / {n}}
\end{equation*}

For Scenario 3 we apply the increase in breeding fail as follows:

\begin{equation*}
{p}_{2}^{{b}_{f}} =1- {\left(1- {p}_{1}^{{b}_{f}} \right)}^{{1} / {n}}
\end{equation*}

\clearpage


\begin{table}
\caption[Notation]{Notation followed to describe the stochastic population model.}\label{tab:tabApp3.1.1}
\begin{tabular}[b]{@{}p{1.5cm}p{13cm}@{}}
\toprule
\textbf{Symbol} & \textbf{Definition}                                                                                                                                                                                             \\ \midrule
$q$   & Number of offsprings per brood in habitat $h$                                                                                                                                                                             \\
$m$   & Number of broods per year                                                                                                                                                                                                 \\
${n}_{Sa}$        & Number of sub-adult stages                                                                                                                                                                                    \\
$x$   & Labels $x$ may take the values $j, sa, b, nb, b_{s}, b_{f}$, where these values refer to juveniles, subadults, adults (i.e. sum of all types), non-breeding adults, successful breeders and failed breeders, respectively \\
$h$   & Index for habitat type, $h=\{1, 2\}$                                                                                                                                                                                      \\
$r$   & Label for subadult stage, $r \in \{r_{1} \cdots r_{n_{Sa}}\}$                                                                                                                                                             \\
$t$   & Subindex for time steps, measured in years, $t= \{1 \cdots 50\}$                                                                                                                                                          \\
${p}_{h}^{b}$     & Probability for an individual to become a breeder (successful or not) in habitat $h$                                                                                                                          \\
${p}_{h}^{b_{f}}$ & Probability for a possible breeder individual to be a failed breeder in habitat $h$                                                                                                                           \\
${p}_{h,S_{x}}$   & Probability of survival in habitat $h$ for individuals $x$                                                                                                                                                    \\
\noalign{\bigskip}
${p}_{1\rightarrow2}^{x}$ \\ ${p}_{2\rightarrow1}^{x}$ & Probability for an adult to move from habitat type 1 to 2, or vice versa                                                                                                 \\
\noalign{\bigskip}
${p}_{1\rightarrow2}^{r}$ \\ ${p}_{2\rightarrow1}^{r}$ & Probability for a stage-r subadult to move from habitat type 1 to 2, or vice versa                                                                                       \\
\noalign{\smallskip}
${J}_{h,t}$       & Number of juveniles that have born in habitat $h$ at time step $t$                                                                                                                                            \\
${S}_{h,t}^{r}$   & Number of stage-$r$ subadults in habitat $h$ in year $t$                                                                                                                                                      \\
${A}_{h,t}$       & Total number of adults in habitat $h$ at time $t$                                                                                                                                                             \\
${A}_{h,t}^{x}$   & Number of adults of type $x$ in habitat $h$ at time step $t$                                                                                                                                                  \\
\noalign{\bigskip}
${\Delta}_{1\rightarrow2}^{x}$ \\ ${\Delta}_{2\rightarrow1}^{x}$   & Number of adults that will move from habitat type 1 to habitat 2, or viceversa                                                                               \\
\noalign{\smallskip}
${\Delta}_{1\rightarrow2}^{r}$ \\ ${\Delta}_{2\rightarrow1}^{r}$   & Number of sub-adults that will move from habitat type 1 to habitat 2, or viceversa                                                                           \\ \bottomrule
\end{tabular}
\end{table}
