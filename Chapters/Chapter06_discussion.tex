%************************************************
\chapter{General discussion and conclusions}\label{ch:discussion}
%************************************************

Life history theory explains how life history traits are selected in concert
due to constraints, trade-offs and correlated selection. The theory has evolved
from focusing on single traits such as clutch size \citep{Lack1946,Skutch1949},
explanations based on the allometric relations among life history traits and
body size \citep{Western1979}, r-K selection where traits are selected
according to the population density, including body size \citep{Pianka1970}, to
age-specific mortality \citep{Gadgil1970,Stearns1976,Charlesworth1980} and
more recently, adding behaviour as a factor interacting with the evolution
of the life history traits \citep{Ricklefs2002,Reale2010a,Sol2016}.
Despite the progress made during all these years, there is a gap in the theory
regarding the effects of the life history on the response of organisms to
environmental changes.
My thesis contributes to the advance of this field in two ways.
First, defining a demographically meaningful axes of life history variation in
birds and confirming the existence of trade-offs among traits and restrictions
for the existing combinations of life history traits (Chapter \ref{ch:LHaxes}).
Second, understanding the mechanisms by which life history interacts with
behaviour and the effects for the species dealing with new environmental
conditions (Chapters \ref{ch:LH-Behaviour model} and \ref{ch:POLS}).

\bigskip

The first objective of this thesis was to describe the diversity of life
history in birds. Despite that it is not the first time that this has been done
\citep{Saether1987,Gaillard1989,Saether2000,Jeschke2009}, my work in
chapter \ref{ch:LHaxes} supposes and advance respect the previous works. I used
newly available statistical methods that control for the phylogenetic structure
of the data by means of phylogenetic principal component analysis
\citep{Revell2009a} and phylogenetic least square regressions \citep{Ho2014}.
Also, the number of species in the compiled dataset is one order of magnitude
larger than previous works. Finally, the availability of demographic data and
the linking of it with different combinations of life history traits allowed to
objectively define the axes of life history variation that better describe
demographic features of the species instead of using traits according to their
availability.
I described the already known fast-slow axis
\citep{Stearns1983a,Saether1987,Gaillard1989,Oli2004,Dobson2007,Jeschke2009},
a iteroparity axis \citep{Gaillard1989}, an axis related to the trade-off
between offspring quantity and quality
\citep{Promislow1990,Bielby2007,Dobson2007}, and finally an axis that reflects
the lifelong productivity of the species in terms of offspring number and egg
mass produced relative to body size.
The availability of a large data base of the species position in each life
history axis opens new possibilities to understand the implications of the life
history in an evolutive and ecological framework.
Life history in birds is not as diverse as in other groups, perhaps due to the
fly constraints \citep{Gaillard1989,Healy2014}.
Future works could apply a similar methodology to more diverse groups such as
mammals for which traits and demographic data is already available
\citep{Myhrvold2015,Salguero-Gomez2016}.

\bigskip

A second objective of the thesis was to explore how life history affects the
response of species to environmental changes. Previous attempts to elucidate
the question show contradictory patterns about the effects of the fast-slow axis
on establishment success of introduced species (non significant for birds
\citep{Blackburn2009a,Sol2012a} but significant for mammals and reptiles
\citep{Capellini2015,Allen2017}), or the ability to colonize urban habitats
(non significant for birds \citep{Sol2014} but significant for mammals
\citep{Santini2019}).
As I argued in chapters \ref{ch:LH-Behaviour model} and \ref{ch:POLS}, one
possible explanation is the omission of an important factor that affects and
is affected by the species position in the fast-slow axis: the behaviour.

As a first approach to the question, I explored mechanisms by which life history
and behavior can interact by means of a theoretical individual based model
simulating the introduction of species with different life histories in a new
environment (Chapter \ref{ch:LH-Behaviour model}). The model shows that under
maladaptive scenarios, where the match of the phenotype to the environment is
insufficient, it pays to have a slow life history that increase the value of
adults over the value of offspring even at the cost of decreasing reproduction.
This is in part owing to the demographic consequences of the life-history
strategy itself and in part owing to the higher benefits of behavioural
responses for slow species in comparison to fast species.
The notion that slow animals exposed to novel environments generally gain
greater benefits from behavioural responses has been suggested in previous
studies (reviewed in \citet{Sol2016}). Animals at the ‘slow’ extreme of the
fast–slow continuum are generally believed to explore more accurately the
environment and exhibit better performance in learning than those at the ‘fast’
extreme, one reason is that they tend to have disproportionally larger brains,
which has been shown to enhance the capacity to innovate and learn
\citep{Lefebvre1997,Reader2002,Overington2009,Reader2011}. The model in chapter
\ref{ch:LH-Behaviour model} supports the idea that life history and behavior are
not independent and that studied together can help to better understand the
mechanisms by which they affect the responses to environmental changes, even
more in the context of human induced rapid environmental changes where
behavioral responses can be the key to adapt to novel conditions. Future studies
can use similar approaches to look for demographic and ecological consequences
of other relevant axes of life history variation such as iteroparity and
offspring quality-quantity as described in Chapter \ref{ch:LHaxes}. In
particular, iteroparity seems a relevant axis to adapt to novel environments as
comparative studies have shown for invasive species \citep{Sol2012a} an urban
dwellers \citep{Sol2014,Sayol2020}.

A second approach to understand how life history and behavior affect the
response to environmental changes was developed in Chapter \ref{ch:POLS}. In
this chapter I used a comparative analysis of flight initiation distances for
birds in rural and urban habitats. The results show the existence of a
peace-of-life syndrome (POLS) predicted by theory where slow-lived species tend
to be more risk-averse than fast-lived species \citep{Reale2010a}.
Furthermore, the POLS structure vanishes in urbanized environments due to
slow-lived species adjusting their flight distances based on the perception of
risk. Even though there is no evidence in birds that slow species are better
urban colonisers \citep{Sol2014}, the fact that slow species have a more plastic
behaviour supports the idea that slow species can potentially better adapt to
to environmental changes.
In the other hand, the pattern seems reverted in mammals, where species with
traits related to the fast end of the fast-slow continuum are better urban
dwellers in some groups \citep{Santini2019}.
The contradictory results about whether fast or slow strategies are better to
deal with environmental changes is still open. The fact that there is no clear
pattern in birds could be because, even though there is a fast-slow continuum
among species, the birds as a group are mostly in the slow end compared to
mammals \citep{Healy2014}.

\bigskip

Environmental changes include a wide range of processes, from land use changes
to introduction of exotic species and climate change. Each type of changes can
affect differently the resource availability and the age-specific mortality,
thus, the effects on the species and their responses or the lack of, should be
different. The nature of the changes will determine if the environment will
become more or less predictable, include novel food resources or qualitatively
different threats such as new predators for which species are not adapted.
For unpredictable environments, bet-hedging strategies can help to survive
\citep{Starrfelt2012}, but also plastic life history strategies mediated by
behavior such as the ability to module the reproductive effort by skipping
reproduction in bad years, the so called ``storage effect'' \citep{Forcada2008}.
Increased juvenile mortality by predators can be compensated by increasing
reproduction if enough resources are available \citep{Yeh2004} displacing the
species towards the fast end of the fast-slow continuum. Also, the increase of
the juvenile survival will have less impact for slow species which exhibit a
life history buffer against the effects of demographic stochasticity and are
less sensitive to changes in juvenile survival as I showed in Chapter
\ref{ch:LH-Behaviour model}.
Species can adapt to predictable changes through adjustments on the life
history \citep{Evans2005} or through behavioral plasticity mechanisms such as
learning \citep{Laundre2001,Evans2012}.
In the case of rapid changes for which there is not enough time for evolutive
responses, behavioral plastic responses may be the only way to adapt to the
new opportunities or threats \citep{Sol2009}, and there is a growing number of
evidences that behavior affects and is affected by life history \citep{Sol2016}.
Thus, when we examine how life history affects the responses to environmental
changes, we are considering not only life history mechanisms but also mechanisms
related to behavioural responses.

Probably, the answer to the question of which life history strategies are
better to respond to environment changes is context-dependent, and only by
carefully thinking about the relevant mechanisms, including behavioral
responses, for each scenario of environmental change and the taxonomic group of
study, we can refine the understanding of the effects of life history on the
ability of the species to survive to environmental changes.


\clearpage

\subpdfbookmark{Conclusions}{Conclusions}
\section*{Conclusions}

\begin{itemize}
  \item \textbf{Chapter 2:}
  \begin{itemize}
    \item Not all combinations of life history traits exist in nature. The
variation of life history traits is constrained and organized in different
axes caused by trade-off and phylogenetic constraints.
    \item One of the axes of traits' covariation is the fast-slow, described
recurrently since 1983 by a varying set of traits more often justified by the
traits data availability than for ecological reasons. This axis is related to
the survival-fecundity trade-off. Defining the fast-slow axis in a
demographically meaningful way (i.e. optimizing the correlation of the
resulting axis with the elasticity to adult survival or generation time)
allowed me to build an objective and more ecologically relevant
characterisation of the axis.
    \item The remaining variation in the life history traits once the fast-slow
is ruled out is organized in three other axes of variation, sorted by variance
explained:

    \begin{itemize}
      \item Iteroparity, describes the degree of concentration of the
reproductive effort in few or many breeding attempts.
      \item Lifelong potential productivity, related to the lifelong investment
in reproduction in terms of the number of offspring and the egg mass production
relative to the body mass.
      \item Offspring quality-quantity trade-off, sorts species along a
continuum with species with a large relative egg mass and small clutch size in
one end, and species with large clutch size and small relative egg size at the
other end.
    \end{itemize}
  \end{itemize}


  \item \textbf{Chapter 3:}
  \begin{itemize}
    \item Theoretical models help to investigate the effects and mechanisms
that affect the demography of species in novel and unknown environments.
    \item Under maladaptive scenarios where the mismatch of phenotype to the
environment is insufficient, simulations suggest that slow lived species,
for which adult have more value than offspring, have more chances to be
established.
    \item Behavioural responses interact with life history to influence the
persistence of populations in novel and unknown environments. The benefits of
learning behaviours are greater for slow strategies. Behaviours such
as skipping a reproductive event, can improve the probabilities to be
established for slow species while being detrimental for fast species. And
finally, innate responses in a context of novel environments can be beneficial
or impact negatively the probabilities to establish a population if preferences
do not match habitat quality (ecological trap), being fast strategies more
impacted by ecological traps on scenarios with higher offspring mortality and
slow strategies more impacted on scenarios with higher adult mortality.
  \end{itemize}


  \item \textbf{Chapter 4:}
  \begin{itemize}
    \item  Slow-lived species tend to have a more risk-averse behaviour than
fast-lived species.
    \item The relationship between flight initiation distance (FID) and the
fast-slow continuum is largely mediated by differences in body size among
species. Possible causes include a higher likelihood to be detected by
predators, lower maneuverability to escape when attacked and higher energetic
costs associated with flight.
    \item Flight initiation distance is shorter in urban than in rural
environments. While slow-lived species showed shorter or larger FID
according to the perception of risk, fast-lived species did not accommodate
their FID to the degree of human frequentation. The changes in FID observed in
slow-lived species may reflect plastic adjustments, selection, and/or a
non-random sorting of individuals by behaviours that affect invasion success.
  \end{itemize}


%   \item \textbf{Chapter 5:}
%   \begin{itemize}
%     \item ...
%     \item ...
%   \end{itemize}
\end{itemize}
